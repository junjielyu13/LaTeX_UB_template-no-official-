% Lengths in LaTex: https://www.overleaf.com/learn/latex/Lengths_in_LaTeX

%% Units:
pt:  %% a point is approximately 1/72.27 inch, that means about 0.0138 inch or 0.3515 mm (exactly point is defined as 1/864 of American printer’s foot that is 249/250 of English foot)
mm:  %%  a millimeter
cm:	 %%  a centimeter
in:	 %%  inch
ex:	 %%  roughly the height of an 'x' (lowercase) in the current font (it depends on the font used)
em:	 %%  roughly the width of an 'M' (uppercase) in the current font (it depends on the font used)
mu:	 %%  math unit equal to 1/18 em, where em is taken from the math symbols family
sp:	 %%  so-called "special points", a low-level unit of measure where 65536sp=1pt


%% Lengths are units of distance relative to some document elements. Lengths can be changed by the command:
\setlength{\lengthname}{value_in_specified_unit}
\baselineskip	%%  Vertical distance between lines in a paragraph 段落中行之间的垂直距离
\columnsep	    %%  Distance between columns                       列之间的距离 
\columnwidth	%%  The width of a column                          列的宽度
\linewidth	    %%  Width of the line in the current environment.           当前环境中线条的宽度。
\evensidemargin	%%  Margin of even pages, commonly used in two-sided documents such as books 偶数页边距,常用于书籍等双面文档
\oddsidemargin	%%  Margin of odd pages, commonly used in two-sided documents such as books 奇数页边距,常用于书籍等双面文档
\paperwidth	    %%  Width of the page                                   页面宽度
\paperheight	%%  Height of the page                                      页面高度
\parindent	    %%  Paragraph indentation                                    段落缩进
\parskip	    %%  Vertical space between paragraphs                       段落之间的垂直间距
\tabcolsep	    %%  Separation between columns in a table (tabular environment) 表中各列之间的分隔(表格环境)
\textheight	    %%  Height of the text area in the page         页面中文本区域的高度
\textwidth      %% 	Width of the text area in the page          页面中文本区域的宽度
\topmargin      %%	Length of the top margin                    上边距的长度